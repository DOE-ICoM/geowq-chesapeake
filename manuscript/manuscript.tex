\documentclass{article}
\usepackage[breaklinks]{hyperref}
\usepackage{graphicx}
\usepackage{placeins}
\usepackage{lipsum}
\usepackage[margin=1in]{geometry}
\usepackage{lineno}
\modulolinenumbers[2]
% \usepackage[authoryear]{natbib}
\usepackage[style=numeric,sorting=none]{biblatex}
\addbibresource{manuscript/geowq.bib}
% \usepackage{draftwatermark}
% \SetWatermarkScale{5}
\usepackage{float}

\title{Geographically aware estimates of remotely sensed water properties for Chesapeake Bay}
\author{Jemma Stachelek$^{1}$, Sofia Avendaño$^{1,}$$^{2}$, Jon Schwenk$^{1}$  \\
        \small $^{1}$Los Alamos National Laboratory, Division of Earth and Environmental Sciences, Los Alamos, NM 87545, USA \\
        \small $^{2}$Nicholas School of the Environment, Duke University, Durham, NC 27708, USA \\
}
\date{}

\begin{document}
\maketitle

\begin{abstract}
    \noindent Remotely sensed water properties are important for a variety of applications, including validation of Earth Systems models (ESMs), habitat suitability models, and sea level rise projections. For the validation of next-generation, high or multi-resolution (30-60 km) ESMs in particular, the usefulness of operational forecasting products and directly-observing satellite-based sensors for validation is limited due to their temporal availability and spatial resolution of $<$ 1 year (in some cases) and $>$ 30 $km^2$ respectively. To address this validation data gap, we developed a data-driven model to produce high-resolution ($<$ 1 $km^2$) estimates of temperature, salinity, and turbidity over decadal time scales as required by next-generation ESMs. Our model fits daily MODIS Aqua reflectance data to surface observations ($<$ 1 m depth) from 2000-2021 in Chesapeake Bay, USA. The resulting models have similar error statistics as prior efforts of this type for salinity (RMSE: 2.2) and temperature (RMSE: 1.9 C). However, unlike prior efforts our model is designed as a pipeline meaning that it has the advantage of producing predictions of water properties in future time periods as additional MODIS data becomes available. We also include novel  “geographically-aware” predictive features insofar as they capture geographic variation in the influence of flow and surface water exchange in upstream coastal watersheds.
    \end{abstract} \hspace{10pt}

Keywords: reflectance, remote sensing, salinity, temperature, turbidity, Chesapeake

\linenumbers

\section{Introduction}

In partially mixed estuaries such as the Chesapeake Bay (hereafter, “the Bay”), productivity, organic matter production, and phytoplankton community composition are largely determined by stratification \cite{xuClimateForcingSalinity2012}. Because stratification is controlled by salinity and temperature, extensive measurement networks exist to monitor these water properties (Chesapeake Bay Program Water Quality Database, \url{https://www.chesapeakebay.net/what/downloads/cbp_water_quality_database_1984_present}). These data are critical for establishing point-based trends and relationships between river flow or sea level rise and salinity or hypoxia \cite{hagyHypoxiaChesapeakeBay2004}. Of particular importance is monitoring for increased nutrient loads as result of expected increases in precipitation, tributary discharge, and river flows as a result of climate change \cite{najjarPotentialClimatechangeImpacts2010, irbyCompetingImpactsClimate2018}.

Unfortunately, direct observations alone are not adequate for many applications that require spatially extensive data such as SLR forecasting and habitat suitability assessments \cite{hoodChesapeakeBayProgram2021}. For example, habitat suitability assessments are typically done on a gridded basis for the entirety of a particular estuary rather than at selected point locations. Applications such as these often turn to gridded alternatives to direct observations such as output from directly-observing satellite-based sensors \cite{foreCombinedActivePassive2016} or operational now/fore-casting products \cite{lanerolle2011second}. Unfortunately, these too are of limited use for applications such as validation of next-generation Earth Systems Models (ESMs, \cite{golazDOEE3SMModel2022}), where the temporal availability of operational forecasting products is too short ($<$ 1 year) and the spatial resolution of directly-observing satellite-based sensors is too coarse ($>$ 30 $km^2$).

To address this validation data gap, we developed a data-driven model to produce high-resolution (< 1 $km^2$) estimates of temperature, salinity, and turbidity at daily time-steps (over decadal time scales) as required by next-generation ESMs. The primary input to our model is remotely sensed reflectance data. In particular, we use Moderate Resolution Imaging Spectroradiometer (MODIS) Aqua multiband data, which has been shown to be predictive of salinity in coastal Louisiana \cite{vogelAssessingSatelliteSea2016, wangDevelopmentMODISData2018}. We supplemented MODIS data with gridded estimates of distance-weighted tributary inflows which we used to encourage physical realism (i.e. geographic “awareness”) especially in areas of the Bay with little to no observations. Our model represents a further advantage over prior efforts at data-driven modeling of salinity in the Bay (e.g. \cite{vogelAssessingSatelliteSea2016, murphyComparisonSpatialInterpolation2010}) because it is calibrated against not only buoy observations taken in the mainstem of the Bay but also observations taken in the mid to lower reaches of Bay tributaries. These more spatially extenstive observations likely increase the accuracy and reliability of model predictions in the upper tributaries.


\section{Methods}

\subsection{Study site}

Chesapeake Bay is located in the Mid-Atlantic region of the United States and serves as the outlet of 9 major tributary systems. Of these, the Susquehanna River (at the Northern extent of the Bay) provides the majority of the freshwater supply to the Bay (\texttildelow$48\%$, \cite{xuClimateForcingSalinity2012}). The Bay is fairly shallow ($<$ 30m, \cite{murphyComparisonSpatialInterpolation2010}) and the hydrodynamics of the Bay are such that it is considered stratified and partially mixed \cite{xuClimateForcingSalinity2012}. For the present study, this means that surface water properties are likely to differ substantially from water properties at deeper depths. It also means that the Bay receives enough freshwater inputs to produce a distinct upper-estuary to lower-estuary gradient in water properties.

\subsection{Model overview}

Prior data-driven models (e.g. \cite{urquhartGeospatialInterpolationMODISderived2013, vogelAssessingSatelliteSea2016}) of Bay water properties used Generalized Linear Models (GLM), Neural Networks (NN), Generalized Additive Models (GAMs), and Random Forests (RF). We chose to use the Random Forest algorithm in part because \cite{urquhartGeospatialInterpolationMODISderived2013} found that it produced nearly equivalent results to more complex methods (particularly NN) and also because of its ease of implementation and interpretation. We fit RF models using the scikit-learn Python package \cite{pedregosaScikitlearnMachineLearning2011} where each model was initialized with 250 trees, had a maximum depth of 20 tree “layers”, and was subjected to recursive feature elimination to produce the most parsimonious model (Figure 1). We subsequently “tuned” each model using k-fold cross validation to optimize the size, depth, and splitting criteria of each tree. We evaluated the “importance” of particular variables as the average impurity decrease of tree splits based on that variable \cite{pedregosaScikitlearnMachineLearning2011}.

\subsection{Data description}

\begin{sloppypar}
We compiled over two decades of observational data (2000-2021) on salinity (n=6k), temperature (n=10k), and turbidity (n=5k, Table S1). Although this data came from a wide variety of sources (n=20, Table S1), almost $80\%$ of the total unique observations came from three programs NDBC/CBIBS (\url{https://buoybay.noaa.gov/}), Maryland Continuous Monitoring Program (MDNR), and the Virginia Estuarine and Coastal Observing System (\url{http://vecos.vims.edu}). The temporal range of the observations (2000-2021) includes the full range of freshwater input magnitudes to the Bay spanning both times of minimum and maximum freshwater input (\url{https://www.usgs.gov/media/images/estimated-annual-mean-streamflow-entering-chesapeake-bay}).
\end{sloppypar}

Because Chesapeake Bay has a mean optical depth of approximately 0.89m \cite{urquhartRemotelySensedEstimates2012}, we filtered all collected observations to those occurring in the top 1m of the water column. We further limited observations to those that had a timestamp during the MODIS overpass window time-of-day (approximately 1700 - 1900 hours GMT). Prior to model fitting, we removed the annual cycle from temperature observations using least squares optimization routines \cite{virtanenSciPyFundamentalAlgorithms2020a}. This annual cycle was added back to model estimates for evaluation purposes and in the production of prediction surfaces. As a final observational data processing step, we implemented spatial and temporal averaging of observational data to a daily mean within each MODIS pixel. 

We supplemented observational data with a grid of “freshwater influence” (FWI) estimates where each grid cell represents the average of annual-flow weighted distance of that cell from the 5 major tributaries (Susquehanna, James, Pautexent, Choptank, and Potomac). Annual flow magnitudes were calculated from National Water Information System instantaneous data (sites: 01488110, 01491000, 01576000, 01654000, 01594440, 02037500, \cite{nwis2021usgs}). Rather than straight-line distances we used “in-water” distances calculated using the scikit-image Python package \cite{littleKrigingEstuariesCrow1997,vanderwaltScikitimageImageProcessing2014}. Our in-water distance calculations involved creating a cost–surface whereby land pixels had a very high cost relative to water pixels. Then the distance from any one cell to a tributary source was the accumulated travel “cost” as given by the scikit-image \texttt{route\_through\_array} function. We combined FWI estimates with the pixel-wise daily mean observations and regressed them in RF models against data from the 500m resolution MODIS Aqua product, which provides 8 bands of values for each pixel spanning from 405 to 877nm wavelength \cite{vermoteericMOD09GAMODISTerra2015}.

All code for data processing, model fitting, and model evaluation is available at Stachelek et al. (2023) the pipeline for which is shown in Figure 1. All data used in the study are available in Stachelek et al. (2023).

\begin{figure}[ht!]
    \begin{center}
          \includegraphics[width=0.75\textwidth,keepaspectratio]{figures/data-processing_model-architecture}
          \caption{Diagram of the data processing, model fitting, tuning, and prediction pipeline.}
    \end{center}    
\end{figure}


\subsection{Model validation}

We validated RF model predictions both quantitatively against a hold-out test set of observational data (one third of the total data) and qualitatively against data from the Chesapeake Bay Operational Forecast System (CBOFS, \cite{lanerolle2011second}). CBOFS provides data on Bay-wide water level, temperature, and salinity every 6 hours. The model itself uses National Oceanic and Atmospheric Administration (NOAA) water level, salinity observations, and US Geological Survey (USGS) river discharge data to calibrate an implementation of the Rutgers University Regional Ocean Modeling System (ROMS, \cite{shchepetkinRegionalOceanicModeling2005}).


\section{Results}

Our tuned RF models for salinity, turbidity, and temperature had a root mean square error (RMSE) of 2.34 ($R^2$=0.91), 32.2 ($R^2$=0.07), and 1.8 ($R^2$=0.96) respectively (Table 1, Figure 2). Out-of-sample validation performance was good for salinity and temperature models but poor for the turbidity model (Figure 2, Figure S1). Despite excellent overall performance for the temperature and salinity models, there was some indication of poor performance at the extremes. The models were unable to recover very low temperature values ($<$ 0 C) or very high temperature and salinity values ($>$ 30 C, $>$ 23 salinity, Figure 2). One explanation for these areas of poor performance is that such extremes were rare in the observational dataset (Table S2). The salinity model however did faithfully capture areas of low salinity (Figure 2, S3). These low salinity values can be regarded as low uncertainty (high confidence) estimates given the extent to which the upper tributaries are well represented in the observational dataset (Figure S2).

\begin{figure}[ht!]
    \begin{center}
          \includegraphics[width=0.75\textwidth,keepaspectratio]{figures/rf_stats_table_1_}          
    \end{center}    
\end{figure}

\begin{figure}[ht!]
    \begin{center}
          \includegraphics[width=0.75\textwidth,keepaspectratio]{figures/_validation}
          \caption{Out of sample test set performance for salinity, turbidity, and temperature models respectively. Hatching indicates the location of 80\% of the data points.}
    \end{center}    
\end{figure}

The date of observation was the most important feature identified in RF training for the temperature and turbidity models while location of observation was most important in the salinity model (Figure 3). Across all models, the most important MODIS band was band 8, which is the shortest wavelength band of the MODIS Aqua product at the extreme of the visible light range. The “size” (i.e. n\_estimators) of the final RF models were of a similar magnitude for each variable but the tree was much “deeper” (i.e. max\_depth) for temperature compared to the other variables (Table S3).

\begin{figure}[ht!]
    \begin{center}
          \includegraphics[width=0.8\textwidth,keepaspectratio]{figures/_importance_all}
          \caption{Random Forest importance plot for salinity, temperature, and turbidity models.}
    \end{center}    
\end{figure}

Running the models in prediction mode produced surfaces with a realistic spatial trend whereby salinity was highest near the Bay mouth and decreased with distance from each tributary mouth (Figure 4). A realistic spatial trend can also be seen in the RF temperature predictions whereby maximum water temperatures were located in the upper tributaries while minimum water temperatures were highest at the Bay mouth (Figure 4).

\begin{figure}[ht!]
    \begin{center}
          \includegraphics[width=0.7\textwidth,keepaspectratio]{figures/_seasonality}
          \caption{Seasonal Random Forest prediction results for temperature and salinity.}
    \end{center}    
\end{figure}

RF model predictions for salinity were generally lower than corresponding values from the CBOFS in the mainstem of the Bay and higher than corresponding CBOFS values in the tributaries (Figure 5). CBOFS surface water salinity estimates are known to be saltier than observations \cite{lanerolle2011second, vogelAssessingSatelliteSea2016} partly explaining our saltier data-driven results. The salinity model was able to reproduce realistic temporal dynamics whereby the Bay is generally “fresher” in the Spring season (April-June) and “saltier” in Fall and Winter seasons (Figure 4).

\begin{figure}[ht!]
    \begin{center}
          \includegraphics[width=0.75\textwidth,keepaspectratio]{figures/_rf-vs-cbofs}
          \caption{Comparison between Random Forest salinity prediction results and a CBOFS snapshot for Sept, 4, 2022.}
    \end{center}    
\end{figure}

\section{Discussion}

Our geographically aware approach was able to accurately reproduce realistic spatial and temporal patterns of Bay salinity and temperature. Model errors were generally lower than prior data-driven efforts and as good or better than physics-based models \cite{vogelAssessingSatelliteSea2016}. In addition to improved accuracy, our approach provides the ability to generate rapid bay-wide prediction surfaces in seconds and it is easily updatable insofar as saved models can be updated as new data become available. Our models are lightweight with minimal internal complexity and have been extensively tested against a uniquely large observational dataset.

Although salinity and temperature models were accurate, our turbidity model showed relatively poor performance. One reason for this may have been because our observational data were reported as Nephelometric Turbidity Units (NTUs) which are not standardized. As a result, they cannot be reliably compared across different instruments. A better turbidity model would likely result from using total suspended matter (TSM) data measured in a standardized unit such as mg/l \cite{ondrusekDevelopmentNewOptical2012}. Another reason the turbidity model performed poorly may be that turbidity is simply difficult to predict. The event-driven nature of turbidity may mean that its extreme spikes may occur over time scales which are too short to be captured by daily reflectance data.
 

\subsection{Spatial patterning}

Temperature and salinity models produced realistic predictions insofar as they were characterized by low salinity in the upper tributaries and high minimum water temperatures at the Bay mouth. This indicates that the model captures both freshwater discharge from tributaries as well as the moderating influence of ocean water in areas near the Bay mouth \cite{dingSpatiotemporalPatternsWater2015}.

Our comprehensive dataset includes greater spatial coverage than prior efforts that were  largely trained on limited (and relatively high-salinity) buoy or in-situ cruise data located in the mainstem of the Bay \cite{vogelAssessingSatelliteSea2016, geigerSatellitederivedCoastalOcean2013, ondrusekDevelopmentNewOptical2012}. This coverage allows for a more comprehensive assessment of the spatial uncertainty of our model. Predicted areas of low salinity coming from our model likely have particularly low associated uncertainties because the upper tributaries were well-represented in our observational dataset (Figure S2). Because CBOFS salinity data was fresher than our RF estimates in the upper tributaries, this suggests that CBOFS data for these areas are likely true underestimates and not a deficiency of the RF model predictions.

In contrast to the upper tributaries, an area of maximum uncertainty is likely the mainstem and eastern Bay across from the mouth of the Potomac (Tangier Sound). This appears to be the collision point of the marine plume emanating from the Bay mouth with the discharge of the Potomac River. The hydrodynamics here seem to be complex given that the CBOFS has difficulty resolving spatial patterning in this area \cite{lanerolle2011second}. This was an area where we found limited observational data (Figure S2). Improvements to data-driven modeling in this area may be possible with the inclusion of more observational data.


\subsection{Temporal patterning}

The salinity model was able to reproduce realistic temporal dynamics whereby the Bay is generally “fresher” in the Spring season (April-June) and “saltier” in Fall and Winter seasons. However, because the salinity model was not able to reproduce the highest salinity values Fall and Winter estimates may be biased low. For the temperature model, we were unable to recover low temperature observations less than 0 deg C. These results are consistent with prior research showing winter had the highest seasonal error  \cite{geigerSatellitederivedCoastalOcean2013}. One explanation for these high winter error rates is the fact that winter in the Bay is generally characterized by strong vertical mixing due to high winds \cite{sonWaterPropertiesChesapeake2012, xuClimateForcingSalinity2012} making “true” patterns more dynamic and more difficult to resolve.


\subsection{Future research}

One shortcoming of our work which may be a fruitful area of future research is the lack of realistic small-scale features evident in the physics based output of the CBOFS (Figure 5). It may be possible to resolve small-scale features such as eddies and other circulation features with the use of convolutional neural networks (CNNs). A key difference between our approach and that of CNNs is that our approach essentially fits each observation in isolation and does not exploit the “spatial locality” (i.e. the “neighborhood”) surrounding observations as is the case with CNNs \cite{goodfellow2016convolutional}. The pipeline for training a CNN would bring some practical difficulties regarding the collection and storage of MODIS data. Unlike our RF approach, it would require obtaining entire multiband images for the Bay rather than simply the values that correspond with observational data.

Another potential area of future research is resolving water properties at depths beyond the surface. A key challenge in this area is lack of observational data at multiple locations. Depth arrayed measurements are typically only associated with buoy deployments of which there are few in the Bay \cite{vogelAssessingSatelliteSea2016}. It may be possible to supplement these buoy observations with data from physics-based models such as CBOFS which are fully three-dimensional. The results of such an effort can be regarded more so as an emulator than a standalone model.


\section{Conclusion}

We have demonstrated an approach to produce realistic estimates of water properties in Chesapeake Bay using remotely sensed data. Salinity and temperature model performance was as good or better than prior efforts. Salinity model performance was particularly realistic in low salinity tributary areas. Much of this realism comes from our FWI routines which encourage physical consistency in upstream-downstream spatial patterning. We saw additional performance gains owing tothe fact that our training dataset had spatial coverage in the upper tributaries beyond the mainstem of the Bay. Estimates produced by our model have a high potential to inform calibration and evaluation of Earth Systems Models as well as in other applications such as habitat suitability and sea level rise models.

\FloatBarrier

\section{Acknowledgements}

J.S. built models, analyzed data, and wrote the paper. S.A. designed the initial model architecture and variable importance routines. J.Sc. conceived the study, collected observational data, and edited the paper. This work was supported by the Earth System Model Development and Regional and Global Modeling and Analysis program areas of the U.S. Department of Energy, Office of Science, Office of Biological and Environmental Research as part of the multi-program, collaborative Integrated Coastal Modeling (ICoM) project. This research used resources provided by the Darwin testbed at Los Alamos National Laboratory (LANL) which is funded by the Computational Systems and Software Environments subprogram of LANL's Advanced Simulation and Computing program (NNSA/DOE).

% \bibliographystyle{apalike}
% \bibliography{manuscript/geowq}
\printbibliography

\end{document}