\documentclass{article}
\usepackage[breaklinks]{hyperref}
\usepackage{graphicx}
\usepackage{placeins}
\usepackage{lipsum}
\usepackage[margin=1in]{geometry}
\usepackage{lineno}
\modulolinenumbers[2]
\usepackage[authoryear]{natbib}
\usepackage{draftwatermark}
\SetWatermarkScale{5}
\usepackage{float}

\title{Geographically aware estimates of remotely sensed water properties for Chesapeake Bay}
\author{Jemma Stachelek$^{1}$, Sofia Avendaño$^{1,}$$^{2}$, Jon Schwenk$^{1}$  \\
        \small $^{1}$Los Alamos National Laboratory, Division of Earth and Environmental Sciences, Los Alamos, NM 87545, USA \\
        \small $^{2}$Nicholas School of the Environment, Duke University, Durham, NC 27708, USA \\
}
\date{}

\begin{document}
\maketitle

\begin{abstract}
    \noindent Remotely sensed water properties are important for a variety of applications, including validation of Earth Systems models (ESMs). However, the usefulness of operational forecasting products and directly-observing satellite-based sensors for validation of next-generation ESMs is limited due to their temporal availability and spatial resolution of $<$ 1 year and $>$ 30 $km^2$ respectively. To address this validation data gap, we developed a data-driven model to produce high-resolution ($<$ 1 $km^2$) estimates of temperature, salinity, and turbidity over decadal time scales as required by next-generation ESMs. Our model fits daily MODIS Aqua reflectance data to surface observations ($<$ 1 m depth) from 2000-2021 in Chesapeake Bay, USA. Our model has similar error statistics as prior efforts of this type for salinity (RMSE: 2.2) and temperature (RMSE: 1.9 C). However, unlike prior efforts our model is set up as a pipeline meaning that it has the advantage of producing predictions of water properties in future time periods as additional MODIS data is collected. In addition, our study is unique in that the predictions produced by our model are “geographically-aware” insofar as they capture geographic variation in the influence of flow and surface water exchange in upstream coastal watersheds.
    \end{abstract} \hspace{10pt}

\linenumbers

\section{Introduction}

In partially mixed estuaries such as the Chesapeake Bay (hereafter, “the Bay”), productivity, organic matter production, and phytoplankton community composition are largely determined by stratification \citep{xuClimateForcingSalinity2012}. Because stratification is primarily controlled by salinity and temperature, extensive measurement networks exist to monitor these water properties (Chesapeake Bay Program Water Quality Database, \url{https://www.chesapeakebay.net/what/downloads/cbp_water_quality_database_1984_present}). These data are critical for many uses such as establishing point-based trends and relationships (e.g. river flow or sea level rise vs salinity or hypoxia), \citet{hagyHypoxiaChesapeakeBay2004}. Of particular importance is monitoring for increased nutrient loads as result of expected increases in precipitation, tributary discharge, and river flows as a result of climate change \citep{najjarPotentialClimatechangeImpacts2010, irbyCompetingImpactsClimate2018}.


Unfortunately, direct observations alone are not adequate for many applications that require spatially extensive data such as SLR forecasting and habitat suitability assessments \citep{hoodChesapeakeBayProgram2021}. For example, habitat suitability calculations are typically done on a gridded basis for the entirety of a particular estuary rather than at selected point locations. Applications such as these often turn to gridded alternatives to direct observations such as output from directly-observing satellite-based sensors \citep{foreCombinedActivePassive2016} or operational now/fore-casting products \citep{lanerolle2011second}. Unfortunately, these too are of limited use for applications such as validation of next-generation Earth Systems Models (ESMs, \citet{golazDOEE3SMModel2022}), where the temporal availability of operational forecasting products is too short ($<$ 1 year) and the spatial resolution of directly-observing satellite-based sensors is too coarse ($>$ 30 $km^2$).

To address this validation data gap, we developed a data-driven model to produce high-resolution (< 1 $km^2$) estimates of temperature, salinity, and turbidity over decadal time scales as required by next-generation ESMs. The primary input to our model is remotely sensed reflectance data. In particular, we use Moderate Resolution Imaging Spectroradiometer (MODIS) Aqua multiband data, which has been shown to be predictive of salinity in coastal Louisiana \citep{wangDevelopmentMODISData2018}. We supplemented MODIS data with gridded estimates of tributary inflows which we used to enforce physical realism (i.e. geographic “awareness”) especially in areas of the Bay with little to no observations. Our model represents a further advantage over prior efforts at data-driven modeling of salinity in the Bay (e.g. \citet{vogelAssessingSatelliteSea2016}) because it is calibrated against not only observations taken in the mainstem of the Bay but also observations taken in the mid to lower reaches of Bay tributaries.


\section{Methods}

\subsection{Study site}

Chesapeake Bay is located in the Mid-Atlantic region of the United States and serves as the outlet of 9 major tributary systems. Of these, the Susquehanna River provides the majority of the freshwater supply to the Bay (\texttildelow$48\%$, \citet{xuClimateForcingSalinity2012}). The Bay is fairly shallow ($<$ 30m, \citet{murphyComparisonSpatialInterpolation2010}) and the hydrodynamics of the Bay are such that it is considered stratified and partially mixed \citep{xuClimateForcingSalinity2012}.

\subsection{Model overview}

Prior data-driven models (e.g. \citet{urquhartGeospatialInterpolationMODISderived2013, vogelAssessingSatelliteSea2016}) of Bay salinity used Generalized Linear Models (GLM), Neural Networks (NN), Generalized Additive Models (GAMs), and Random Forests (RF). We chose to use the Random Forest algorithm in part because \citet{urquhartGeospatialInterpolationMODISderived2013} found that it produced nearly equivalent results to more complex methods (particularly NN) but also because of its ease of implementation and interpretation. We fit RF models using the scikit-learn Python package \citep{pedregosaScikitlearnMachineLearning2011} where each model was initialized with 250 trees, had a maximum depth of 20 tree “layers”, and was subjected to recursive feature elimination to produce the most parsimonious model. We subsequently “tuned” each model using k-fold cross validation to optimize the size and splitting criteria of each tree. We evaluated the “importance” of a particular variable as the average impurity decrease of tree splits based on that variable \citep{pedregosaScikitlearnMachineLearning2011}.

\subsection{Data description}

We compiled over two decades of observational data (2000-2021) on salinity (n=6k), temperature (n=10k), and turbidity (n=5k, Table S1). Although this data came from a wide variety of sources (n=20, Table S1), almost $80\%$ of the total unique observations came from three programs NDBC/CBIBS (\url{https://buoybay.noaa.gov/}), Maryland Continuous Monitoring Program (MDNR), and the Virginia Estuarine and Coastal Observing System (\url{http://vecos.vims.edu}). Because Chesapeake Bay has a mean optical depth of 0.89m \citep{urquhartRemotelySensedEstimates2012}, we filtered all collected observations to those occurring in the top 1m of the water column. We further limited observations to those that had a timestamp during the MODIS overpass window time-of-day (approximately 1700 - 1900 hours). Prior to model fitting, we removed the annual cycle from temperature observations using least squares optimization routines \citep{virtanenSciPyFundamentalAlgorithms2020a}. Finally, we implemented spatial and temporal averaging of observational data to a daily mean within each MODIS pixel. 

We supplemented observational data with a grid of “freshwater influence” (FWI) estimates where each grid cell represents the average of annual-flow weighted distance of that cell from the 5 major tributaries (Susquehanna, James, Pautexent, Choptank, and Potomac). Annual flow magnitudes were calculated from National Water Information System instantaneous data (sites: 01488110, 01491000, 01576000, 01654000, 01594440, 02037500, \citet{nwis2021usgs}). Rather than straight-line distances we used “in-water” distances calculated using the scikit-image Python package \citep{vanderwaltScikitimageImageProcessing2014}. We combined FWI estimates with the pixel-wise daily mean observations and regressed them in RF models against data from the 500m resolution MODIS Aqua product, which provides 8 bands values for each pixel spanning from 405 to 877nm wavelength \citep{vermoteericMOD09GAMODISTerra2015}.


\subsection{Model validation}

We validated RF model predictions both quantitatively against a hold-out test set of observational data (one third of the total data) and qualitatively against data from the Chesapeake Bay Operational Forecast System (CBOFS, \citet{lanerolle2011second}). CBOFS provides data on Bay-wide water level, temperature, and salinity every 6 hours. The model itself uses National Oceanic and Atmospheric Administration (NOAA) water level, salinity observations, and US Geological Survey (USGS) river discharge data to calibrate an implementation of the Rutgers University Regional Ocean Modeling System (ROMS, \citet{shchepetkinRegionalOceanicModeling2005}).


\section{Results}

Tuned RF models for salinity, turbidity, and temperature had a root mean square error (RMSE) of 2.34 ($R^2$=0.91), 32.2 ($R^2$=0.07), and 1.8 ($R^2$=0.96) respectively (Table 1, Figure S1). Out of sample validation performance was excellent for salinity and temperature models but poor for the turbidity model (Figure 1). Despite excellent overall performance, our temperature and salinity models had poor performance at the extremes. They were unable to recover very low values ($<$ 0 C) and very high values ($>$ 30 C, $>$ 23 salinity, Figure 1). These areas of poor performance can be explained by their relative rarity in the observational dataset (Table S2).

\begin{figure}[ht!]
    \begin{center}
          \includegraphics[width=0.75\textwidth,keepaspectratio]{figures/rf_stats_table_1_}          
    \end{center}    
\end{figure}

\begin{figure}[ht!]
    \begin{center}
          \includegraphics[width=0.75\textwidth,keepaspectratio]{figures/_validation}
          \caption{Out of sample test set performance for salinity, turbidity, and temperature models respectively. Hatching indicates the location of 80\% of the data points.}
    \end{center}    
\end{figure}

The date of observation was the most important feature in RF training for the temperature and turbidity models while location of observation was most important in the salinity model (Figure 2). Across all models, the most important MODIS band was band 8. The “size” (i.e. n\_estimators) of the final RF models were of a similar magnitude for each variable but the tree was much “deeper” (i.e. max\_depth) for temperature compared to the other variables (Table S3).

\begin{figure}[ht!]
    \begin{center}
          \includegraphics[width=0.8\textwidth,keepaspectratio]{figures/_importance_all}
          \caption{Random Forest importance plot for salinity, temperature, and turbidity models.}
    \end{center}    
\end{figure}

Running the models in prediction mode produced surfaces with a realistic spatial trend whereby salinity was highest near the Bay mouth and decreased with distance from each tributary mouth (Figure 3). RF model predictions were generally lower than corresponding values from the CBOFS in the mainstem of the Bay and higher than corresponding CBOFS values in the tributaries (Figure 4). The salinity model was able to reproduce realistic temporal dynamics whereby the Bay is generally “fresher” in the Spring season (April-June) and “saltier” in Fall and Winter seasons (Figure 2).

\begin{figure}[ht!]
    \begin{center}
          \includegraphics[width=0.4\textwidth,keepaspectratio]{figures/_seasonality}
          \caption{Seasonal Random Forest salinity prediction results.}
    \end{center}    
\end{figure}

\begin{figure}[ht!]
    \begin{center}
          \includegraphics[width=0.75\textwidth,keepaspectratio]{figures/_rf-vs-cbofs}
          \caption{Comparison between Random Forest salinity prediction results and a CBOFS snapshot for Sept, 4, 2022.}
    \end{center}    
\end{figure}

\section{Discussion}

Our data-driven yet geographically aware approach was able to accurately reproduce spatial and temporal patterns of Bay salinity and temperature. In addition to being accurate, our approach provides the ability to generate rapid bay-wide prediction surfaces in seconds and it is easily updatable insofar as saved models can be reloaded and have their training “updated”. Our models are also lightweight with minimal internal complexity and extensively tested against a uniquely large observational dataset.

\subsection{Spatial patterning}

\subsection{Temporal patterning}

\subsection{Future research}

One shortcoming of our work which may be a fruitful area of future research is the lack of realistic small-scale features evident in the physics based output of the CBOFS (Figure 4). These small-scale features such as eddies and other circulation features may be resolved with the use of convolutional neural networks (CNNs). A key difference between our approach and that of CNNs is that our approach fits each observation essentially in isolation and does not attempt to reproduce the “neighborhood” surrounding observations as is the case with CNNs (citation). 

Another potential area of future research is resolving water properties at depths beyond the surface. A key challenge in this area is lack of observational data at multiple locations. Depth arrayed measurements are typically only associated with buoy deployments of which there are few in the Bay. Perhaps these can be supplemented by data from physics-based models such as CBOFS. The results of such an effort can be regarded more so as an emulator than a standalone model.

\FloatBarrier
\bibliographystyle{apalike}
\bibliography{manuscript/geowq}

\end{document}